%%%%%%%%%%%%%%%%%%%%%%%%%%%%%%%%%%%%%%%%%%%%%%%%%%%%%%%%%%%%%%%%%%%
%% 
%% Xianwei Zhang's cv
%%  -  credit: Yisong Yue, http://www.yisongyue.com/resume/
%%   - based off work by Michael DeCorte 
%%
%%%%%%%%%%%%%%%%%%%%%%%%%%%%%%%%%%%%%%%%%%%%%%%%%%%%%%%%%%%%%%%%%%%



%%
%% The following code sets up the document formatting
%%

%this assumes that xwz_cv.sty is in some path
%%XW: in .sty file, I added two parts, 'contact' and 'people'
%%XW: differing from 'documentclass' (\usepackage{xx}), packages used by '\documentstyle' are provided in the brackets
%%XW: xcolor, package to define colors
%%XW: marvosym, package for symbols (letter, cell, mouse and writinghand, etc)
%%XW:  marvosym: http://texdoc.net/texmf-dist/doc/fonts/marvosym/marvodoc.pdf
%%XW: more fancy symbols can be found in fontawesome: http://mirror.jmu.edu/pub/CTAN/fonts/fontawesome/doc/fontawesome.pdf
%%XW: for 'fontawesome', package 'fontspec' and typeset 'LuaLatex' should be used
\documentstyle[hyperref, margin, line, xcolor, marvosym]{xwz_cv}

\hypersetup{backref,pdfpagemode=Full,colorlinks=true,backref}

%%% XW: http://timmurphy.org/2009/07/22/adjusting-latex-margins/
\addtolength{\oddsidemargin}{-0.45in}
\addtolength{\hoffset}{0.30in}		%the horizontal offset of the text (how fast the text is from the left of the page)
\addtolength{\voffset}{-0.60in}		%the vertical offset of the text (how fast the text is from the top of the page)
\addtolength{\textwidth}{0.80in} 	%the width of the text on the page
\addtolength{\textheight}{0.80in}	%the height of the text on the page

\renewcommand{\namefont}{\LARGE\emph}

%%
%% The following code defines some macros for terms which have raised font
%% (ie 4\fourth would result 4th with the 'th' raised (superscripted)
%%

\def\Cplusplus{{\rm C\raise.5ex\hbox{\small ++}}}
\def\CSharp{{\rm C\raise.5ex\hbox{\small \#}}}
% 'st' 'nd' 'rd' 'th' superscripts for numbers
\def\first{{\raise.5ex\hbox{\small st}}}
\def\second{{\raise.5ex\hbox{\small nd}}}
\def\third{{\raise.5ex\hbox{\small rd}}}
\def\fourth{{\raise.5ex\hbox{\small th}}}

%% my commands
\newcommand{\mysidestyle}{\sc}						%%XW: side format, small capitals
\newcommand{\me}[1]{{\tt #1}}							%%XW: hightlight me as author
\newcommand{\pub}[1]{\em\textbf{#1}}					%%XW: pub title format
\newcommand{\tlk}[1]{\em{#1}}							%%XW: talk title format
%%XW: original colors, white, black, red, green, blue, cyan, magenta, yellow
%%XW: see https://en.wikibooks.org/wiki/LaTeX/Colors
\definecolor{oxfordblue}{rgb}{0.0, 0.13, 0.28}				%%XW: define color, http://latexcolor.com/
\newcommand{\web}[2]{\href{#1}{\color{oxfordblue} { {#2}}}{}}	%%hyper link format	, arg1: url, arg2: word to display
   
%%
%% starting the actual document
%%

\begin{document}

\topmargin 0.1in									%%XW: adjust the margin to top

%the name in big fonts at the top of resume
%this is left aligned
\name{Xianwei ZHANG}
%%XW: 'contact' is contact info, at the same line with name, i.e., above the horizontal line
%%XW: not auto right aligned, please use '\hskip' instead
\contact{\hskip 3.1cm  \Letter \web{xianeizhang@cs.pitt.edu}{xianeizhang@cs.pitt.edu} \hskip 2.6cm {\Large \ComputerMouse}\web{http://iarchsys.com}{http://iarchsys.com}}

%this is right aligned
\address{
\Mobilefone \ +1 412-425-4362 \ \  {\large \WritingHand} 6514 Sennott Square, Pittsburgh PA-15206
}

\begin{resume}

\vspace{-6mm}
\section{\mysidestyle Research\\ Interests}
\textbf{Memory System}, Computer Architecture and Systems\\
My research interests lie broadly in computer architecture and system with particular emphasis
on memory system design and optimization on the critical aspects of latency, energy and
bandwidth. Besides memory design, I also explore memory topics in extended scenarios such
as approximate computing and security.\\
In addition, I also have broad interests on other system areas/topics, e.g. operating system,
GPGPU and compilation, which are partially covered in my\web{http:iarchsys.com}{tech blog} and\web{https://people.cs.pitt.edu/~xianeizhang/\#NOTE}{notes}.

%%%%
%%%%XW: generally, for each section, format should be first defined
%%%%XW: '\body', i.e., '\desc' cannot be left out
%%%%

%%
%% EDUCATION
%%
%\vspace{+3mm}
\section{\mysidestyle{Education}}

\newcommand{\university}{\organization}			%univesity
\newcommand{\degree}{\title}					%degree
%\newcommand{\thesis}{\employer}				%thesis
\begin{formatb}
  \degree{l}\dates{r}\\							%degree year
  \university{l}\\								%university
 % \body									%desc
\end{formatb}

\degree{\textbf{Ph.D. candidate in Computer Science}} \dates{Aug 2011 - Apr 2017 (expected)}
\university{University of Pittsburgh, Pittsburgh, USA}
\begin{desc}
  \begin{itemize}
  	\item {\tt Dissertation Topic}: \textit{"Exploration of DRAM Scaling from Restoring Perspective"}
	\item {\tt Advisor}:  Prof.\web{http://people.cs.pitt.edu/~zhangyt/}{\em{Youtao Zhang}}
	\item {\tt Co-advisors} (working with): Prof.\web{http://www.pitt.edu/~juy9/}{\em{Jun Yang}} (ECE) and Prof.\web{http://people.cs.pitt.edu/~childers}{\em{Bruce Childers}} (CS)
  \end{itemize}
\end{desc}

\degree{\textbf{B.E. in Software Engineering}} \dates{Sep 2007 - Jun 2011}
\university{Northwestern Polytechnical University, Xi'an, China}
\begin{desc}								%%XW: leave 'desc' blank
\end{desc}

%%
%% WORK EXPERIENCE
%%
\section{\mysidestyle{Experiences}}

\begin{formatb}
  \organization{l}\title{r}\\
  \location{l}\dates{r}\\
  \body\\
\end{formatb}

\organization{\textbf{University of Pittsburgh}}	\title{Graduate Student Researcher}
\location{Pittsburgh, USA}					\dates{May 2013 -- Present}
\begin{desc}
  \vspace{-0.1in}
  \begin{itemize}
  	\item Dec 15- Now\ , Apply approximate computing to achieve energy-accuracy tradeoff.
	\item Jan 14-Sep 15, Mitigate performance and yield issues in further scaling DRAM.
	\item Oct 14-Mar 15, Improve bandwidth and performance in Hybrid Memory Cube (HMC).
	\item Feb 13-May 15, Construct energy efficient non-volatile memories (PCM and DWM).
  \end{itemize}
\end{desc}

\organization{\textbf{University of Pittsburgh}}	\title{Teaching Assistant}
\location{Pittsburgh, USA}					\dates{Aug 2011 -- Apr 2013}
\begin{desc}
CS0401 (Fall'11, Spring'12), CS1502 (Fall'12), CS2520 (Spring'13), CS2210 (Spring'15)
\end{desc}

\organization{\textbf{Alipay Technology Inc.}, Alibaba}	\title{Java Developer Intern}
\location{Hangzhou, China}				\dates{Aug 2010 -- Dec 2010}
\begin{desc}
Implemented source management system based on SOFA/Spring framework.
\end{desc}


%%
%% PUBLICATIONS
%%
\section{\mysidestyle{Publications}}

\newcommand{\authors}{\people}					%authors
\newcommand{\conf}{\organization}					%conf
\begin{formatb}
  \authors{l} \conf{r}\\
  \body\\
\end{formatb}

\authors{\me{Xianwei Zhang}, Youtao Zhang, Bruce R. Childers and Jun Yang} 		\conf{HPCA'2016}
\begin{desc}
  {\pub{Restore Truncation for Performance Improvement in Future DRAM Systems}.} \\
  The 22nd IEEE Symposium on High Performance Computer Architecture(HPCA), Barcelona, Spain, 2016.
\end{desc}

\authors{\me{Xianwei Zhang}, Youtao Zhang, Bruce R. Childers and Jun Yang} 		\conf{TODAES}
\begin{desc}
  {\pub{On the Restore Time Variations of Future DRAM Memory}}, ({\em under review}).\\
  ACM Transactions on Design Automation of Electronic Systems (TODAES).
\end{desc}

\authors{\me{Xianwei Zhang}, Youtao Zhang, Bruce R. Childers and Jun Yang} 		\conf{DATE'2015}
\begin{desc}
  {\pub{Exploiting DRAM Restore Time Variations in Deep Sub-micron Scaling}.}\\
  The IEEE conference on Design, Automation and Test in Europe (DATE), Grenoble, France, 2015.
\end{desc}

\authors{\me{Xianwei Zhang}, Youtao Zhang and Jun Yang} 					\conf{ICCD'2015}
\begin{desc}
  {\pub{DLB: Dynamic Lane Borrowing for Improving Bandwidth and Performance in Hybrid Memory Cube}.}\\
  The 33rd IEEE International Conference on Computer Design(ICCD), NYC, NY, 2015.
\end{desc}

\authors{\me{Xianwei Zhang}, Youtao Zhang and Jun Yang} 					\conf{ICCD'2015}
\begin{desc}
  {\pub{TriState-SET: Proactive SET for Improved Performance in MLC Phase Change Memories}.}\\
  The 33rd IEEE International Conference on Computer Design(ICCD), NYC, NY, 2015.
\end{desc}

\authors{\me{Xianwei Zhang}, Youtao Zhang and Jun Yang} 					\conf{ICCD'2015}
\begin{desc}
  {\pub{Exploit Common Source-Line to Construct Energy Efficient Domain Wall Memory based Caches}.}\\
  The 33rd IEEE International Conference on Computer Design(ICCD), NYC, NY, 2015.
\end{desc}

\authors{\me{Xianwei Zhang}, Youtao Zhang and Jun Yang} 					\conf{DAC'2015}
\begin{desc}
  {\pub{Adaptive Lane Borrowing of Hybrid Memory Cube}}, ({\em Work-in-progress}).\\
  The 52nd ACM/IEEE Design Automation Conference (DAC), San Francisco, CA, June 2015.
\end{desc}

\authors{\me{Xianwei Zhang}, Youtao Zhang, Chuanjun Zhang and Jun Yang} 		\conf{ISLPED'2013}
\begin{desc}
  {\pub{WoM-SET: Lowering Write Power of Proactive-SET based PCM Write Strategy Using WoM Code}.}\\
  The International Symposium on Low Power Electronics and Design (ISLPED), Beijing, China, 2013. \\
  \textbf{$\star\star\star$ Best Paper Award $\star\star\star$}
\end{desc}

%%
%% Presentations
%%
\section{\mysidestyle{Presentations}}
	
\newcommand{\talktype}{\organization}				%type of presentation
\newcommand{\dateloc}{\location}					%dateloc
\begin{formatb}
  \talktype{l} \dateloc{r}\\
  \body\\
\end{formatb}

\talktype{{\em HPCA} Symposium}					\dateloc{{\em Mar 2016}, Barcelona, Spain}
\begin{desc}
	\tlk{Restore Truncation for Performance Improvement in Future DRAM Systems}
\end{desc}

\talktype{Thesis Proposal}							\dateloc{{\em Jan 2016}, Pittsburgh, USA}
\begin{desc}
	\tlk{Exploration of DRAM Scaling from Restoring Perspective}
\end{desc}

\talktype{{\em ICCD} Symposium}					\dateloc{{\em Oct 2015}, New York City, USA}
\begin{desc}
	\tlk{DLB: Dynamic Lane Borrowing for Improving Bandwidth and Performance in Hybrid Memory Cube\\ 
	TriState-SET: Proactive SET for Improved Performance of MLC Phase Change Memories}
\end{desc}

\talktype{{\em MemSys} Symposium}				\dateloc{{\em Oct 2015}, Washington DC, USA}
\begin{desc}
	\tlk{Achieving Yield, Density and Performance Effective DRAM at Extreme Technology Sizes}
\end{desc}

\talktype{CompExam}							\dateloc{{\em Jul 2015}, Pittsburgh, USA}
\begin{desc}
	\tlk{Shared Resources Management and Execution Replay in Chip Multiprocessor}
\end{desc}

%%
%% HONORS
%%
\section{\mysidestyle{Honors \&\\ Awards}}

\newcommand{\prize}{\title}						%prize
\newcommand{\issuedby}{\organization}				%issuedby
\begin{formatb}
  \prize{l}	\issuedby{r}\\
  %\body\\
\end{formatb}

\prize{\textbf{Andrew Mellon Predoctoral Fellowship}}					\issuedby{{\em University of Pittsburgh}'2016}
\begin{desc}
\end{desc}

\prize{Student Travel Awards}										\issuedby{{\em HPCA}'2016, {\em SPAA}'2015}
\begin{desc}
\end{desc}

\prize{\textbf{Best Paper Award}}									\issuedby{{\em ISLPED}'2013}
\begin{desc}
\end{desc}

\prize{Recipient of 2011 graduation design (Thesis) key support fund}		\issuedby{{\em NPU}'2011}
\begin{desc}
\end{desc}

\prize{National Scholarship}										\issuedby{{\em Ministry of Education of China}'2010}
\begin{desc}
\end{desc}

\prize{\textbf{Tencent Technology Excellence Scholarship}}				\issuedby{{\em Tencent Inc.}'2009}
\begin{desc}
\end{desc}

%%
%% SKILLS, Nothing special here, just a normal table
%%

\section{\mysidestyle Skills}
 \noindent\begin{tabular}{@{}lll}
 \textbf{Programming:}   	& \ \ & C/C++, JAVA, Shell, Python, Android \\ 
 \textbf{Simulation:}   	& \ \ & GEM5, MARSSx86, DRAMSim2, USIMM, CACTI, SPICE \\ 
 \textbf{Tools:}   		& \ \ & Pin, Vim, Makefile, Git, GDB, \LaTeX, gcc/g++ \\
  \end{tabular}

%%
%% MISC
%%
\section{\mysidestyle{Misc}}

 \noindent\begin{tabular}{@{}lll}		%%XW: cancel the table indent
 \textbf{Homepage:\ \ \ }   	& \ \ & \web{https://people.cs.pitt.edu/~xianeizhang/}{https://people.cs.pitt.edu/$\sim$xianeizhang/} \\ 
 %\textbf{Linkedin:}   		& \ \ & \web{www.linkedin.com/in/xianweizhang}{www.linkedin.com/in/xianweizhang} \\ 
 \textbf{Github:}   		& \ \ & \web{https://github.com/cinwell}{https://github.com/cinwell} \\
 \textbf{Blog:}   			& \ \ & \web{http://iarchsys.com}{http://iarchsys.com} \\
  \end{tabular}

%%
%% REFERENCES
%%
\section{\mysidestyle{References}}
Available upon request.

\end{resume}
\end{document}

